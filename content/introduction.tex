\chapter{Introduction}%(WIP)
\label{chapter:introduction}

In the digital realm it is often trivial to make endless copies of information, with equally endless alterations. For a currency to exist digitally and be widely adopted, its users must believe its supply is strictly limited. A money recipient must trust they are not receiving counterfeit coins, or coins that have already been sent to someone else. To accomplish these goals without requiring the collaboration of any third party like a central authority, the currency's supply and complete transaction history must be publicly verifiable.

We can use cryptographic tools to allow data registered in an easily accessible database --- the blockchain --- to be virtually immutable and unforgeable, with legitimacy that cannot be disputed by any party.
\\ \newline
Cryptocurrencies typically store transactions in the blockchain, which acts as a public ledger\footnote{In this context ledger just means a record of all currency creation and exchange events. Specifically, how much money was transferred in each event and to whom.} of all the currency operations. Most cryptocurrencies store transactions in clear text, to facilitate verification of transactions by the community of users.
\\ \newline
Clearly, an open blockchain defies any basic understanding of privacy or fungibility\footnote{``\textbf{Fungible} means capable of mutual substitution in use or satisfaction of a contract. A commodity or service whose individual units are so similar that one unit of the same grade or quality is considered interchangeable with any other unit of the same grade or quality. Examples: tin, grain, coal, sugar, money, etc." \cite{mises-org-fungible} In an open blockchain such as Bitcoin, the coins owned by Alice can be differentiated from those owned by Bob based on the `transaction history' of those coins. If Alice's transaction history includes transactions related to supposedly nefarious actors, then her coins might be `tainted' \cite{bitcoin-big-bang-taint}, and hence less valuable than Bob's (even if they own the same amount of coins). Reputable figures claim that newly minted Bitcoins trade at a premium over used coins, since they don't have a history \cite{new-bitcoin-premium}.}, since it literally {\em publicizes} the complete transaction histories of its users.
\\ \newline
To address the lack of privacy, users of cryptocurrencies such as Bitcoin can obfuscate transactions by using temporary intermediate addresses \cite{DBLP:journals/corr/NarayananM17}. However, with appropriate tools it is possible to analyze flows and to a large extent link true senders with receivers \cite{DBLP:journals/corr/ShenTuY15b, DK-police-tracing-btc, Andrew-Cox-Sandia, chainalysis-2020-report}.

In contrast, the cryptocurrency MobileCoin attempts to tackle the issue of privacy by storing only single-use addresses for ownership of funds in the blockchain, authenticating the dispersal of funds in each transaction with ring signatures, and verifying transactions within black-box `secure enclaves' that discard extraneous information after verification. With these methods there are no known effective ways to link receivers or trace the origin of funds.\footnote{Depending on the behavior of users, if an attacker manages to break into secure enclaves there may be cases where transactions can be analyzed to some extent. For an example see this article: \cite{monero-ring-heuristics-ryo}.}

Additionally, transaction amounts in the MobileCoin blockchain are concealed behind cryptographic constructions, rendering currency flows opaque even in the case of secure enclave failures.

The result is a cryptocurrency with a high level of privacy and fungibility.



\section{Objectives}
\label{sec:goals}

MobileCoin is a new cryptocurrency employing a novel combination of techniques. Many of its aspects are either backed by technical documents missing key details pertinent to MobileCoin, or by non-peer-reviewed papers that are incomplete or contain errors.\footnote{Seguias has created the excellent Monero Building Blocks series \cite{monero-building-blocks}, which contains a thorough treatment of the cryptographic security proofs used to justify Monero's signature schemes. Seguias's series is focused on v7 of the Monero protocol, however much of what he discusses is applicable to MobileCoin, which uses many of the same building blocks as Monero.} Other aspects can only be understood by examining the source code and source code documentation (comments and READMEs) directly.

Moreover, for those without a background in mathematics, learning the basics of elliptic curve cryptography, which MobileCoin uses extensively, can be a haphazard and frustrating endeavor.

We intend to address this situation by introducing the fundamental concepts necessary to understand elliptic curve cryptography, reviewing algorithms and cryptographic schemes, and collecting in-depth information about MobileCoin’s inner workings.

To provide the best experience for our readers, we have taken care to build a constructive, step-by-step description of the MobileCoin cryptocurrency.

In the first edition of this report we have centered our attention on version 1 of the MobileCoin protocol\footnote{The `protocol' is the set of rules that each new block is tested against before it can be added to the blockchain. This set of rules includes the `transaction protocol' (currently version 1, which we call {\tt TXTYPE\_RCT\_1} for clarity), which are general rules pertaining to how a transaction is constructed.}, corresponding to version 1.0.0 of the MobileCoin software suite. All transaction and blockchain-related mechanisms described here belong to those versions.\footnote{The MobileCoin codebase's integrity and reliability is predicated on assuming enough people have reviewed it to catch most or all significant errors. We hope that readers will not take our explanations for granted, and verify for themselves the code does what it's supposed to. If it does not, we hope you will make a responsible disclosure (by emailing \url{security@mobilecoin.foundation}) for major problems, or Github pull request (\url{https://github.com/mobilecoinfoundation/mobilecoin}) for minor issues.}%[[[revisit]]]\footnote{Several protocols worth considering for the next generation of MobileCoin transactions are undergoing research and investigation, including Triptych \cite{triptych-preprint}, RingCT3.0 \cite{ringct3-preprint}, Omniring \cite{omniring-paper}, and Lelantus \cite{lelantus-preprint}.}



\section{Readership}

We anticipate many readers will encounter this report with little to no understanding of discrete mathematics, algebraic structures, cryptography\footnote{An extensive textbook on applied cryptography can be found here: \cite{applied-cryptography-textbook}.}, or blockchains. We have tried to be thorough enough that laypeople with a diversity of backgrounds may learn about MobileCoin without needing external research.

We have purposefully omitted, or delegated to footnotes, some mathematical technicalities, when they would be in the way of clarity. We have also omitted concrete implementation details where we thought they were not essential. Our objective has been to present the subject half-way between mathematical cryptography and computer programming, aiming at completeness and conceptual clarity.\footnote{Some footnotes, especially in chapters related to the protocol, spoil future chapters or sections. These are intended to make more sense on a second read-through, since they usually involve specific implementation details that are only useful to those who have a grasp of how MobileCoin works.}



\section{Origins of the MobileCoin cryptocurrency}

MobileCoin's whitepaper was released in November 2017 by Joshua Goldbard and Moxie Marlinspike. According to the paper, their motivation for the project was to ``develop a fast, private, and easy-to-use cryptocurrency that can be deployed in resource constrained environments to users who aren't equipped to reliably maintain secret keys over a long period of time, all without giving up control of funds to a payment processing service." \cite{mobilecoin-whitepaper}%[[[update whitepaper citation when rehosted]]]

\if 0
%%%%%%%%%%%%%%%%%%%%%%%%%%%%%%%%%%%%%%%%%%%%%%%%%%%%%%%%%%%%%%%%%%%%%%%
\subsection{Transaction protocol}
\label{subsec:origins-tx-protocol}

To address the design goal of private transactions, MobileCoin adopted a transaction protocol inspired by the cryptocurrency Monero's {\tt RCTTypeBulletproof2} transaction type.

Monero, initially known as BitMonero, was created in April 2014 as a derivative of the proof-of-concept currency CryptoNote \cite{bitmonero-launched}.

CryptoNote is a cryptocurrency devised by various individuals. A landmark whitepaper describing it was published under the pseudonym of Nicolas van Saberhagen in October 2013 \cite{cryptoNoteWhitePaper}. It offered receiver anonymity through the use of one-time addresses, and sender ambiguity by means of ring signatures.

Over time, Monero enhanced the original protocol by implementing the amount hiding technique described by Greg Maxwell (among others) in \cite{Signatures2015BorromeanRS} as integrated into ring signatures based on Shen Noether's recommendations in \cite{MRL-0005-ringct}, then making it more efficient with Bulletproofs \cite{Bulletproofs_paper}. Those features are present in MobileCoin as well.


\subsection{Consensus protocol}

MobileCoin also embraced the Stellar consensus protocol for consentuating transactions. [[[history of stellar consensus protocol]]]


\subsection{}

Finally, MobileCoin makes extensive use of SGX secure enclaves, 
\fi
%%%%%%%%%%%%%%%%%%%%%%%%%%%%%%%%%%%%%%%%%%%%%%%%%%%%%%%%%%%%%%%%%%%%%%%


\section{Outline}

Since MobileCoin is a very new cryptocurrency, we will not go into detail on hypothetical second-layer extensions and applications of the core protocol in this edition. Suffice it to say for now that topics like multisignatures, transaction proofs, and escrowed marketplaces are just as feasible for MobileCoin as they are for any well-designed progeny of the CryptoNote protocol.\footnote{Monero, initially known as BitMonero, was created in April 2014 as a derivative of the proof-of-concept currency CryptoNote \cite{bitmonero-launched}. Subsequent changes to Monero's transaction type retained parts of the original CryptoNote design (specifically, one-time addresses, ring signatures of one form or another, and key images). This means CryptoNote is in some sense an ancestor of MobileCoin, whose transaction scheme is inspired by Monero's {\tt RCTTypeBulletproof2} transaction type.}


\subsection{Essentials}%[[[revisit for proper references]]]

In our quest for comprehensiveness, we have chosen to present all the basic elements of cryptography needed to understand the complexities of MobileCoin, and their mathematical antecedents. In Chapter \ref{chapter:basicConcepts} we develop essential aspects of elliptic curve cryptography.

Chapter \ref{chapter:advanced-schnorr} expands on the Schnorr signature scheme introduced in the prior chapter, and outlines the ring signature algorithms that will be applied to achieve confidential transactions. Chapter \ref{chapter:addresses} explores how MobileCoin uses addresses to control ownership of funds, and the different kinds of addresses.

In Chapter \ref{chapter:pedersen-commitments} we introduce the cryptographic mechanisms used to conceal amounts. Chapter \ref{chapter:membership-proofs} is dedicated to membership proofs, which are used to prove MobileCoin transactions spend funds that exist in the blockchain. With all the components in place, we explain the transaction scheme used by MobileCoin in Chapter \ref{chapter:transactions}.

%\ref{chapter:enclaves}
We shed light on secure enclaves in Chapter 8, the MobileCoin blockchain is unfolded in Chapter~9, and the MobileCoin consensus protocol used to create the blockchain is elucidated in Chapter 10.%\ref{chapter:consensus}


\subsection{Extensions}

A cryptocurrency is more than just its protocol. As part of MobileCoin's original design, a `service-layer' technology known as Fog was developed. Discussed in Chapter 11, Fog is a service that searches the blockchain and identifies transaction outputs owned by its users. This allows users to avoid scanning the blockchain themselves, which is time-consuming and resource-intensive (a prohibitive burden for e.g.\ mobile devices). Importantly, the service operator is not able to learn more than approximately how many outputs its users own.%\ref{chapter:fog}


\subsection{Additional content}
%\ref{appendix:RCTTypeBulletproof2}
%\ref{appendix:block-content}
%\ref{appendix:origin-block}
Appendix A explains the structure of a sample transaction that was submitted to the blockchain. Appendix B explains the structure of blocks in MobileCoin's blockchain. Finally, Appendix C brings our report to a close by explaining the structure of MobileCoin's origin (a.k.a.\ genesis) block. These provide a connection between the theoretical elements described in earlier sections with their real-life implementation.

We\marginnote{Isn't this useful?} use margin notes to indicate where MobileCoin implementation details can be found in the source code.\footnote{Our margin notes are accurate for version 1.0.0 of the MobileCoin software suite, but may gradually become inaccurate as the codebase is constantly changing. However, the code is stored in a git repository (\url{https://github.com/mobilecoinfoundation/mobilecoin}), so a complete history of changes is available.} There is usually a file path, such as transaction/std/src/transaction\_builder.rs, and a function, such as \(\textrm{{\tt create\_output()}}\). Note: `-' indicates split text, such as Ristretto- Point $\rightarrow$ RistrettoPoint, and we neglect namespace qualifiers (e.g.\ {\tt TransactionBuilder::}) in most cases.

Some code references are to third-party libraries, which we tagged with square brackets. These include
\begin{itemize}
    \item {[dalek25519]}: the {\tt dalek-cryptography curve25519-dalek} library \cite{dalek-curve25519-lib}
    \item {[dalekEd]}: the {\tt dalek-cryptography ed25519-dalek} library \cite{dalek-ed25519-lib}
    \item {[dalekBP]}: the {\tt dalek-cryptography bulletproofs} library \cite{dalek-bulletproofs-lib}
    \item {[blake2]}: the Rust implementation \cite{blake2-rust-lib} of hashing algorithm Blake2 \cite{blake-hashing-algorithm}
\end{itemize}

To shorten their length, margin notes related to the transaction protocol use the tag [MC-tx], which stands for the directory path `transaction/core/'. Code from an internal code base at MobileCoin (which has not yet been made open source) is tagged with [MC-prop]. Finally, functions that are executed exclusively within SGX secure enclaves are marked with an asterisk `*'.


\section{Disclaimer}

All signature schemes, applications of elliptic curves, and implementation details should be considered descriptive only. Readers considering serious practical applications (as opposed to a hobbyist's explorations) should consult primary sources and technical specifications (which we have cited where possible). Signature schemes need well-vetted security proofs, and implementation details can be found in the source code. In particular, as a common saying among cryptographers and security engineers goes, ``don't roll your own crypto''. Code implementing cryptographic primitives should be well-reviewed by experts and have a long history of dependable performance.\footnote{Cryptographic primitives are the building blocks of cryptographic algorithms. For example, in elliptic curve cryptography the primitives include point addition and scalar multiplication on that curve (see Chapter \ref{chapter:basicConcepts}).} %Moreover, original contributions in this document may not be well-reviewed and are likely untested, so readers should exercise their judgement when encountering them.



\section{History of `Mechanics of MobileCoin'}

`Mechanics of MobileCoin: First Edition' is an adaptation of the public domain `Zero to Monero: Second Edition', published in April 2020 \cite{ztm-2}. `Zero to Monero' itself (its first edition was published in June 2018 \cite{ztm-1}) is an expansion of Kurt Alonso's master's thesis, `Monero - Privacy in the Blockchain' \cite{kurt-original}, published in May 2018.

There are several notable differences between `Mechanics of MobileCoin: First Edition' and `Zero to Monero: Second Edition'.

\begin{itemize}
    \item Parts of the core content were improved, such as an updated group theory section (Chapter \ref{chapter:basicConcepts}) and better narrative flow in the chapter on transactions (Chapter \ref{chapter:transactions}), or adjusted for MobileCoin, such as the addition of the Ristretto abstraction (Chapter \ref{chapter:basicConcepts}).
    \item Transaction details specific to MobileCoin replaced details specific to Monero (for example, the addition of membership proofs in Chapter \ref{chapter:membership-proofs}, which serve a purpose similar to Monero's output offsets).
    \item MobileCoin's consensus protocol (an implementation of the Stellar Consensus Protocol \cite{stellar-consensus-protocol}) replaced Monero's more standard mining-based (Nakamoto \cite{Nakamoto_bitcoin}) consensus mechanism (Chapter 10).
    \item MobileCoin uses secure enclaves extensively (Chapter 8). They have an important role in MobileCoin's privacy model and are essential to the novel Fog technology (Chapter 11).%[[[fix references]]]\ref{chapter:enclaves}\ref{chapter:fog}
\end{itemize}



\section{Acknowledgements}
\label{sec:acknowledgements}

%Written by author `koe'.

This report, like `Zero to Monero' before it, would not exist without Alonso's original master's thesis \cite{kurt-original}, so to him I (koe) owe a great debt of gratitude. Robb Walters got me involved with MobileCoin, and is without doubt a force for good in this crazy world. All I can feel is admiration for the team at MobileCoin, who have designed and built something that goes far beyond what anyone could reasonably hope for in a cryptocurrency. Finally, it is hard to express in words how incredible the legacy of modern technology is. MobileCoin would truly be impossible without the prior research and work of countless people, only a tiny subset of whom can be found in this document's bibliography.