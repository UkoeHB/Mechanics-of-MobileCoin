% this file is called up by main.tex
% content in this file will be fed into the main document

% ---------------------------------------------------------------------------

Cryptography. It may seem like only mathematicians and computer scientists have access to this obscure, esoteric, powerful, elegant topic. In fact, many kinds of cryptography are simple enough that anyone can learn their fundamental concepts.
\\ \newline
It is common knowledge that cryptography is used to secure communications, whether they be coded letters or private digital interactions. Another application is in so-called cryptocurrencies. These digital moneys use cryptography to assign and transfer ownership of funds. To ensure that no piece of money can be duplicated or created at will, cryptocurrencies usually rely on `blockchains' containing public, distributed ledgers with records of currency transactions that can be verified by third parties \cite{Nakamoto_bitcoin}.
\\ \newline
It might seem at first glance that transactions need to be sent and stored in plain text format to make them publicly verifiable. In truth, it is possible to conceal a transaction's participants, as well as the amounts involved, using cryptographic tools that nevertheless allow transactions to be verified and agreed upon by observers \cite{cryptoNoteWhitePaper}. This is exemplified in the cryptocurrency MobileCoin.
\\ \newline
We endeavor here to teach any determined individual who knows basic algebra and simple computer science concepts like the `bit representation' of a number not only how MobileCoin works at a deep and comprehensive level, but also how useful and beautiful cryptography can be.
\\ \newline
For our experienced readers: MobileCoin is a standard one-dimensional directed acyclic graph (DAG) cryptocurrency blockchain \cite{Nakamoto_bitcoin}, where blocks are consensuated with an implementation of the Stellar Consensus Protocol \cite{stellar-consensus-protocol}, transactions are validated in SGX secure enclaves \cite{intel-sgx-explained-advanced} and are based on elliptic curve cryptography using the Ristretto abstraction \cite{ristretto} on curve Ed25519 \cite{Bernstein2012-high-speed-high-security-ed25519}, transaction inputs are shown to exist in the blockchain with Merkle proofs of membership \cite{merkle-tree} and are signed with Schnorr-style multilayered linkable spontaneous anonymous group signatures (MLSAG) \cite{MRL-0005-ringct}, and output amounts (communicated to recipients via ECDH \cite{Diffie-Hellman}) are concealed with Pedersen commitments \cite{maxwell-ct-2} and proven in a legitimate range with Bulletproofs \cite{Bulletproofs_paper}.