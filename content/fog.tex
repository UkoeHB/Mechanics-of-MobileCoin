\chapter{Fog Service}
\label{chapter:fog-service}

As discussed in Section \ref{subsec:subaddresses-sending-to}, for Bob to identify outputs he owns on the blockchain, he must perform (at least \cite{reduce-scan-times-view-tag-research-issue-73}) a Diffie-Hellman exchange on every txout public key that exists (compute $k_B^v*r_t K_t^{s,i}$). Moreover, this implies obtaining (e.g.\ downloading) a copy of every transaction output. In terms of bandwidth and computational effort, using this method to identify owned outputs is prohibitively expensive for weak devices with bandwidth constraints (e.g.\ mobile devices or hardware wallets)~\cite{mobilecoin-whitepaper}.

`Fog' is a service designed to address those problems by offloading the cost of scanning the blockchain to external servers, so users of the service can learn about their owned outputs efficiently. A naive approach would be for users to give copies of their private view keys to the service operator, however that would allow the operator to learn extensive information about the users' financial activities.

Instead, a Fog service is able to identify outputs owned by its users, and pass them along to those users, without the operator learning much more than the approximate number of outputs involved. To accomplish that, Fog employs a combination of SGX enclaves and an ORAM (oblivious memory) data structure called an OMAP (oblivious map; see the implementation in \cite{mobilecoin-omap-source-code}).\footnote{The content in this chapter was sourced from \cite{privacy-properties-mobilecoin-fog}, \cite{mobilecoin-fog-rfc-fog-without-user-table}, and the MobileCoin team's implementation of Fog (actively in development and not released into open source at the time of writing this). Note that Fog could be abstracted away from MobileCoin and implemented in a generic way. We limit our exploration to the only existing implementation, which is MobileCoin-centric.}%[[[need citation for oram]]]



\section{Fog-enabled addresses}
\label{sec:fog-enabled-addresses}

At its core, the Fog system has two steps. First, a Fog provider identifies outputs owned by its users (Section \ref{sec:fog-processing-blockchain}). Second, its users obtain copies of those outputs (Section \ref{sec:fog-obtaining-outputs}).

Fog providers identify outputs by looking at an {\em encrypted fog hint} stored in each output. A fog hint is simply the output recipient's public view key encrypted with a Fog `ingress key' (Section \ref{sec:fog-sending-outputs-alice-to-bob}). Ingress keys can change from time to time, so if Alice wants to send an output to Bob, she must contact his desired Fog provider to obtain an active/current key.

To that end, a standard address in MobileCoin includes identifying information about the address-holder's desired Fog provider. Such an address is formatted\marginnote{account- keys/src/ account\_ keys.rs {\tt {\em struct} PublicAdd- ress}} like this:\vspace{.155cm}%[[[needs margin note]]]
\[\textrm{[public view key][public spend key][fog url][fog report id][fog signature]}\]

The Fog-related fields are as follows.
\begin{itemize}
    \item \textbf{fog url}: An identifier for locating and contacting the Fog service, along with the method for locating it (e.g.\ the http/https protocols). It has the format `method://identifier'.\footnote{A URL (Uniform Resource Locator) is a {\em locator} that tells you how to locate a resource. URLs are a subset of so-called URIs (Uniform Resource Identifiers). \cite{uri-vs-url-semantics}}

    \item \textbf{fog report id}: If an address-holder's Fog provider is operating multiple distinct {\tt fog-ingest} enclaves, this identifier indicates which enclave's ingest key to use when constructing an output for the address-holder.\footnote{As we will see, there isn't much point for a Fog provider to have multiple {\tt fog-ingest} enclaves, so in practice most Fog providers are likely to only use one `fog report id'. However, the id may allow some providers to offer `early access' to Fog service upgrades. Early access users could test new Fog features or infrastructure before they are rolled out to the main user-base.}

    \item \textbf{fog signature}: A `root' key belonging to the Fog provider is signed with the address-holder's public view key. As we will see, this helps ensure encrypted fog hints are only readable by Bob's chosen provider even in the case of a secure enclave breach.
\end{itemize}

How Bob chooses his Fog provider and obtains the Fog components of his address are implementation details not pertinent to this discussion.%[[[would be nice to have citations pointing to documentation on these details]]]



\section{Sending outputs from Alice to Bob}
\label{sec:fog-sending-outputs-alice-to-bob}

For Alice to send an output to Bob via Fog she must add an encrypted fog hint to the output. That hint is Bob's public view key encrypted with a Fog ingress key. Clearly if his Fog provider knows the private ingress key they can trivially learn which outputs are owned by Bob.

Rather, the private ingress key lives in a {\tt fog-ingest} SGX secure enclave. To give transaction authors confidence they are only creating encrypted fog hints readable by a secure enclave, fog ingress keys sent to authors must have corresponding {\tt Attestation Verification Reports} ({\tt AVR}s, recall Section \ref{subsec:enclaves-sgx-attestation}).\footnote{Alice could conspire with Bob's Fog provider to use a non-enclave key for the encrypted fog hint. However, Alice has plenty of other ways to reveal Bob as a recipient of her transaction, so conspiring in that way is not a specific threat Bob should worry about. The bigger danger is Alice's wallet software leaking information to the Fog provider or another party, unbeknownst to Alice or Bob.}


\subsection{Ingress keys}
\label{subsec:fog-ingress-keys}

Ingress keys, denoted $K_{ingress} = k_{ingress} G$, are Ristretto points generated\marginnote{*[MC-fog] fog/ingest/ enclave/ impl/src/ lib.rs {\tt enclave\_ init()}}[-.8cm] when a {\tt fog-ingest} enclave is initialized. The private ingress keys are encrypted and stored locally for re-use when an enclave is restarted.\footnote{Specifically, private ingress keys are encrypted with an SGX Seal key\marginnote{*sgx/service/ src/lib.rs {\tt seal\_data()}} using {\tt KEYPOLICY.MRENCLAVE} set to {\tt true} (recall Table \ref{table:egetkey-key-variant-patterns} footnote \ref{footnote:sgx-seal-key-keypolicy}), so only enclaves on the same machine with the same measurement can decrypt it.} They can also be transmitted to `back-up' enclaves, which are `peer' enclaves with the same {\tt MRENCLAVE} value that can be activated in case the primary enclave has a problem.%[MC-[prop] src/fog/ingest/server/src/server.rs sync_key_from_primary() for peering the ingress key

The {\tt fog-ingest} sub-service (i.e.\ the host process that owns the enclave) cooperates with its enclave to produce an {\tt AVR} using $K_{ingress}$ as the report content\marginnote{sgx/report- cache/untru- sted/src/ lib.rs {\tt start\_rep- ort\_cache()}}. It is realistic to assume SGX secure enclaves can and will be breached (even if infrequently, with problems repaired quickly), so ingress {\tt AVR}s are signed with a key belonging to the Fog service operator. This Fog key is in turn signed with another Fog key, and so on in a `certificate chain' up to a so-called `certificate root key'. The `fog signature' in Bob's address is a signature on that root key.\footnote{A certificate chain is used instead of a single Fog service key for easier key management. The root key can be stored in a very secure location, while intermediate keys and final signing keys can be managed in more convenient ways, and discarded easily without invalidating users' fog signatures (which should have an indefinite shelf-life).}

Since enclaves are periodically updated, ingress keys and their {\tt AVR}s become invalid as time passes. As such, `{\tt pubkey\_expiry}' values are attached to ingress {\tt AVR}s (but not part of the report content).\footnote{Ingress key {\tt pubkey\_expiry}s are defined by the Fog operator and can be altered at any time, although they should only ever be {\em increased}, as decreasing them may cause tombstone blocks set before the decrease to be so high that a transaction is added to the blockchain after an ingress key has expired and been discarded.} A {\tt pubkey\_expiry} is the anticipated block height where the ingress key will expire. Practically speaking, this means each transaction's tombstone block should be less than the lowest {\tt pubkey\_expiry} of its Fog-using recipients' ingress keys\marginnote{transaction/ std/src/ transaction\_ builder.rs {\tt impose\_to- mbstone\_bl- ock\_limit()}}.

The ingress key {\tt AVR}, {\tt AVR} Fog signature, certificate chain, and {\tt pubkey\_expiry} are stored in anticipation of requests from transaction authors.


\subsection{Acquiring an ingress key}
\label{subsec:fog-acquiring-incress-key}

Alice can use the `fog url' in Bob's address to contact his Fog provider.\footnote{`Self-spends', or outputs that are addressed to oneself, are treated, in terms of Fog\marginnote{mobile- coind/src/ payments.rs {\tt build\_tx\_ proposal()}}, the same as outputs addressed to others. In other words, if Alice sends an output to herself (e.g.\ a change output), she will contact her own Fog provider with the same procedure used to contact Bob's provider, and encrypted fog hint construction proceeds in the same way as well.} She receives a set of ingress key {\tt AVR}s (with associated material) for all the {\tt fog-ingest} enclaves being run by the provider.\footnote{Within the Fog service is a so-called {\tt fog-report} sub-service which keeps track of {\tt AVR}s and passes them along to requesters.} It turns out those {\tt AVR}s also have a `fog report id' attached (matching of {\tt AVR} to `fog report id' is up to the Fog provider's discretion). Alice chooses\marginnote{fog/report/ validation/ src/lib.rs {\tt get\_fog\_ pubkey()}}[-.6cm] the {\tt AVR} corresponding to the `fog report id' in Bob's address.\footnote{It would seem like the `fog url' could be used to communicate which {\tt AVR} the Fog service should send Alice. However this would leak to the Fog provider some information about Alice's intended recipient.}\footnote{The Fog provider may be able to maliciously provide Bob with a unique fog report id and somehow leverage it to learn information about the outputs he owns. One way to detect this would be creating a bunch of new Fog addresses and seeing if they are all assigned the same (small) set of fog report ids.}

By verifying the {\tt AVR}, the {\tt AVR} Fog signature, the certificate chain, and Bob's address's fog signature, Alice can be sure the {\tt AVR}'s ingress key was either produced by a secure enclave or at the very least belongs to Bob's chosen Fog provider (who he presumably trusts to some extent).


\subsection{Encrypted fog hints}
\label{subsec:fog-encrypted-fog-hints}

Once Alice has a Fog ingress key $K_{ingress}$, creating the encrypted fog hint is straightforward\marginnote{[MC-tx] src/fog\_ hint.rs {\tt encrypt()}}.

\begin{enumerate}
    \item Generate a random scalar $r \in_R \mathbb{Z}_l$.
    \item Compute
    \begin{align*}
        r&*G \\
        r&*K_{ingress}
    \end{align*}
    \item Encrypt Bob's view key $K^v_B$ with $r K_{ingress}$\footnote{\label{footnote:fog-hint-encryption-scheme}Encrypted fog hints rely in practice on a custom encryption scheme, described in \cite{mobilecoin-custom-payload-encryption}, that is similar to ECIES \cite{practical-cryptography-ecies-scheme} but implemented for the Ristretto abstraction. Basically the scheme\marginnote{crypto/box/} takes a message $\mathfrak{m}$ and public key $K$ as inputs, generates a temporary key $r$, produces a shared secret $r K$, uses the shared secret to encrypt the message, creates a hash (known as a `MAC') of the message and shared secret (which can be used to check if message decryption succeeded), and outputs $r G$, the MAC, and the encrypted message.}\vspace{.155cm}
    \[\textrm{\tt enc}[r K_{ingress}](K^v_B)\]
    \item Assemble the full encrypted fog hint\footnote{An encrypted fog hint is 82 bytes, which includes the 32 byte key $r G$, 48 bytes for the encrypted view key (16 bytes for the MAC, 32 bytes for the encrypted text), and 2 bytes of padding that aren't used in the initial version of Fog.}\vspace{.155cm}
    \[\{r G, \textrm{\tt enc}[r K_{ingress}](K^v_B), \textrm{2 bytes padding}\}\]
\end{enumerate}

If Alice wants to send an output to Carol, but Carol doesn't have a Fog-enabled address, then Alice can make a fake encrypted hint so Carol's output appears the same as all other outputs. One approach\marginnote{[MC-tx] src/encryp- ted\_fog\_ hint.rs {\tt fake\_one- time\_hint()}} is to generate a random fake ingress key and encrypt an empty byte-field.\footnote{As noted in footnote \ref{footnote:address-wallet-standards} of Chapter \ref{chapter:addresses}, there is no absolute requirement for wallet implementations to conform with transaction-construction standards like one-time addresses or encrypted fog hints. In the case of one-time addresses, non-conforming wallets are likely to create outputs completely unusable by other wallets. However, it may be possible to design wallets that eschew support for Fog, since the blockchain can always be scanned in the normal fashion. Such wallets should reject attempts to send funds to Fog-enabled addresses, to ensure they only create outputs for recipients that don't use/rely on Fog.}\footnote{\label{footnote:fog-hint-janus-mitigation}The encrypted fog hint provides a convenient way to mitigate the Janus attack (recall Chapter \ref{chapter:addresses} footnote \ref{footnote:janus-attack}). If instead of randomly generating the txout private key $r_t$ it was set equal to the encrypted hint's $r$ value, then MobileCoin users could compute $k^v(k^s + \mathcal{H}_n(k^v,i))*r G \stackrel{?}{=} r_t K^{s,i}_t$ to check if the txout public key provided in a given output they receive was produced by the same subaddress spend key that was used to create the output's one-time address $K^o_t$. This approach has not been implemented in any form at the time of writing this.}



\section{Processing the blockchain}
\label{sec:fog-processing-blockchain}

A Fog provider identifies its users' outputs by passing all transaction outputs from the blockchain into a {\tt fog-ingest} enclave, which attempts to decrypt the fog hints. That enclave produces records of each output that can be easily recovered by their owners.


\subsection{Decrypting fog hints}
\label{subsec:fog-decrypting-hints}

On the surface, identifying Fog users' outputs seems fairly trivial. Transaction outputs are passed into a {\tt fog-ingest} enclave, which uses its private ingress key $k_{ingress}$ to decrypt the outputs' encrypted fog hints. If decryption succeeds (i.e.\ the hint's MAC matches a value computed from the decrypted message) and the decrypted value is a valid Ristretto point, then that point must be the output recipient's public view key.\footnote{Technically any public key whose private key is known by the recipient can be used in fog hints, however using the public view key is a convention set by the initial implementation of Fog.}

However, decrypting hints takes place in a secure enclave owned by a perhaps-not-trustworthy Fog operator. That operator has full visibility on the material passed in and out of an enclave, and the timing of those events. If there is an observable difference in how outputs are treated (i.e.\ between outputs that are and aren't owned by Fog users), then the Fog operator can use that difference as the basis for analyzing his users' behavior \cite{privacy-properties-mobilecoin-fog}.\footnote{Performing analyses based on evidence gained from indirect observation of a computer process is known as a `side channel' attack \cite{nist-side-channel-attack-def}. If outputs can be categorized by Fog, then a Fog operator who learns the identity of a transaction's author (e.g.\ by tracing the IP address used to submit a transaction to the network) may gain insight into that author's recipients. Other insights may be possible as well, but by eliminating the side channel itself their pursuit is rendered futile.}

To prevent the decryption of fog hints from leaking any timing information, they are decrypted\marginnote{(E) [MC-tx] src/fog\_ hint.rs {\tt ct\_decr- ypt()}} in `constant time'. This means the time it takes the decryption algorithm to execute is not a function of the inputs passed in (i.e.\ it's a `data oblivious' algorithm \cite{data-oblivious-data-structures}).


\subsection{Making outputs discoverable}
\label{subsec:fog-making-outputs-discoverable}

It is not realistic to maintain ingress keys indefinitely. In the first place, as noted in Chapter \ref{chapter:enclaves} footnote \ref{footnote:enclave-secret-migration-dangerous}, enclave secrets can't be safely migrated between enclave versions, while the SGX enclave technology receives periodic updates. This means once the owner of an output has been identified, the output needs to be stored in a permanent form so it can be recovered at any time (or even multiple times if necessary).

The basic solution adopted is for each output to be encrypted using the decrypted fog hint key (usually the recipient's public view key by convention), and emitted from the {\tt fog-ingest} enclave tagged with a `fog search key' (based on an `egress key') that the recipient can recompute privately later on (Section \ref{sec:fog-obtaining-outputs}).

\subsubsection{Egress keys}

Each time a {\tt fog-ingest} enclave is started up it generates\marginnote{*[MC-fog] fog/ingest/ enclave/ impl/src/ lib.rs {\tt new()}}[-.75cm] a unique `egress key' $k_{egress}$,\footnote{`Ingress keys' process Fog inputs (outputs from the chain), while `egress keys' help create the {\tt ETxOutRecords} emitted from Fog.} which is just an Ed25519 private key, and sends out a so-called {\tt RngRecord} to its host process for permanent storage. The record\marginnote{[MC-fog] fog/sql\_re- covery\_db/ src/lib.rs {\tt new\_ingest\_ invoca- tion()}}[.8cm] contains $k_{egress}*G$, the first block processed by the enclave after that egress key was created, and the current `rng algorithm' used to create fog search keys.

Egress keys live shorter lives than ingress keys. In particular, this is because they must be recreated each time an enclave is started up (that can't be saved or sent to peer enclaves).\footnote{If an egress key could be reused by multiple enclave instances, then it could be used as part of a replay attack by the enclave operator. The operator could feed fake outputs addressed to Bob to a hidden enclave in order to see what fog search keys are produced, then watch the fog search keys emitted by a real {\tt fog-ingest} enclave (using the same egress key) to figure out which real on-chain outputs are owned by Bob (the operator can correlate outputs sent into a {\tt fog-ingest} enclave with {\tt ETxOutRecord}s sent out from the enclave).} They are also occasionally recreated within a given enclave instance.

\subsubsection{Fog search keys}

After a fog hint has been decrypted, the {\tt fog-ingest} enclave creates\marginnote{*[MC-fog] fog/ingest/ enclave/ impl/src/ lib.rs {\tt attempt\_in- gest\_txs()}} a shared secret between the decrypted key and the egress key. Since this must be constant time, if decryption fails the enclave will make a shared secret with a randomly generated key instead (it is generated before decrypting the fog hint so its creation can't be used for timing analysis). The final shared secret we denote $k_{egress}*K_{hint}$, where $K_{hint}$ could be a real or fake fog hint key.

To accommodate users who may receive an arbitrary number of outputs to the same hint key, fog search keys depend on a counter that increments each time the fog-ingest enclave encounters a given successfully-decrypted hint key. These counters are maintained in an ORAM hash map (similar to a table) called an OMAP, which is owned by the {\tt fog-ingest} enclave.

OMAPs are designed so accessing and modifying entries is constant time.\footnote{We will not explore OMAPs in any depth. Suffice it to say for our purposes that an OMAP is a hash map that can be accessed and modified in constant time. Moreover, the structure that MobileCoin uses employs an encryption scheme so that OMAPs created by an enclave can use memory outside the {\tt ELRANGE}, enabling arbitrarily large structures to be set up and used without leaking any information to the enclave owner. The implementation is open sourced at \cite{mobilecoin-omap-source-code}.} {\tt Fog-ingest} enclaves construct\marginnote{ *[MC-fog] fog/ingest/ enclave/ impl/src/ rng\_store.rs {\tt new()}}[-.75cm] a new OMAP for counters each time a new egress key is created. The shared secret $k_{egress}*K_{hint}$ is used to locate counters in the map, so even if a {\tt fog-ingest} enclave operator manages to read the contents he won't necessarily learn much about output recipients.

Using the counter value obtained from the map (it is a fake/arbitrary value if $K_{hint}$ is fake), an output's fog search key $FSK$ is simply\marginnote{[MC-fog] fog/kex\_rng/ src/versio- ned/kexrng- 20201124.rs {\tt prf()}}:\footnote{The fog search key hash includes the `fog search algorithm' version. This may not be strictly necessary, however it forces wallet developers to keep track of changes to the Fog search key algorithm (which may in reality never change).}\footnote{The hash algorithm used outputs 64 bytes, but only 16 bytes are considered necessary for fog search keys to be secure.}\footnote{Since fog search keys are based directly off the shared secret between an egress key and view key by convention, it is not possible to create a $FSK$ chain that includes outputs from multiple subaddresses (note that a non-standard implementation could use the same fog hint key for multiple subaddresses). In other words, Fog is a service that only collects outputs discoverable by individual addresses and subaddresses by default.}\vspace{.155cm}
\[FSK = \mathcal{H}(\textrm{algo version}, k_{egress}*K_{hint}, \textrm{counter})\textrm{.first\_16\_bytes}\]

The counter is incremented and put back in the table (only if $K_{hint}$ is real), or a new OMAP entry is added if $K_{hint}$ hasn't been encountered before (again, only if $K_{hint}$ is real).\footnote{Note that this means the counter OMAP can only contain entries corresponding to unique keys found while a given egress key is active.}\footnote{If the counter OMAP gets full then no more unique entries can be added. In that scenario the existing OMAP is deleted, a new one is set up, and the {\tt fog-ingest} enclave creates a new egress key (and emits it in an {\tt RngRecord}). This strategy is functionally equivalent to restarting the enclave.} In this way all of a user's outputs' fog search keys created while an egress key is active can be recomputed from the user's private view key, the public egress key, and incrementing a counter from 0.

\subsubsection{Encrypted output records}

To prepare an output for easy use by its owner\marginnote{*[MC-fog] fog/ingest/ enclave/ impl/src/ lib.rs {\tt attempt\_in- gest\_txs()}}[-.3cm], a {\tt TxOutRecord} structure is created. That structure contains the output (sans encrypted fog hint), its on-chain index (i.e.\ its index in all the {\tt TxOut}s ever created), its block's index, and that block's timestamp (to indicate when the output was created). The {\tt TxOutRecord} is encrypted with the same scheme used to encrypt fog hint keys (recall footnote \ref{footnote:fog-hint-encryption-scheme}). This time the encryption key is $K_{hint}$ so the recipient can decrypt it later (Section~\ref{sec:fog-obtaining-outputs}).

The fog search key $FSK$ is placed alongside that encrypted structure in an {\tt ETxOutRecord}, which is emitted\marginnote{[MC-fog] fog/sql\_rec- overy\_db/ src/lib.rs {\tt add\_block\_ data()}} from the {\tt fog-ingest} enclave for permanent storage. One {\tt ETxOutRecord} is created for every {\tt TxOut} sent into the enclave, so the enclave operator cannot discern which ones belong to individuals using his service, and which ones do not.


\subsection{Missed blocks}
\label{subsec:fog-missed-blocks}

Astute readers may have noticed a critical hole in the Fog system as described. An output can only be discoverable if (A) Alice submits it to the network with a fog hint encrypted by an active ingress key and (B) that fog hint is decrypted by the {\tt fog-ingest} enclave using that ingress key. If requirement (A) or (B) isn't met, the {\tt fog-ingest} enclave will output a junk {\tt ETxOutRecord}.

It is ultimately Alice's responsibility to make sure outputs she creates can be found by her intended recipients, so requirement (A) is out of scope for the design of Fog (aside from facilitating ingress keys with reasonable {\tt pubkey\_expiry}s; also, see footnote \ref{footnote:fog-ingest-enclave-version-transition}).

Normally, {\tt fog-ingest} enclaves actively scan new outputs\marginnote{[MC-fog] fog/ingest/ server/src/ controller.rs {\tt process\_ne- xt\_block()}} as they are added to the blockchain, so if Alice submits a transaction soon after acquiring an ingress key, the {\tt fog-ingest} enclave that processes it should use the appropriate private ingress key. However, this is only in the ideal scenario. If the enclave crashes, its machine dies, or there is a Fog service-wide outage, then Alice's transaction may not be properly examined. These potentialities are handled with a defense-in-depth approach.\footnote{\label{footnote:fog-ingest-enclave-version-transition}To transition between enclave versions, the following procedure is feasible: a) start a new-version {\tt fog-ingest} enclave (it generates a new $K_{ingress}$), b) stop making the old ingress key available to transaction authors (i.e.\ its {\tt AVR}) and start making the new one available, c) once the old ingress key's highest {\tt pubkey\_expiry} is reached (the Fog service provider might choose to periodically raise the expiry block of active ingress keys, so only the highest expiry should be used) shut down the old-version {\tt fog-ingest} enclave. For wallets to properly verify new {\tt AVR}s, they likely have to be updated with each update to Fog enclaves. A similar procedure can be used to rotate ingress keys (i.e.\ deprecate an old key and start using a new one). Rotation can be useful for reducing the cost of a given ingress key being compromised (i.e.\ reduce the number of outputs' fog hints a single ingress key can decrypt).}

\begin{enumerate}
    \item Basic lifetime: Ingress keys are generated\marginnote{*[MC-fog] fog/ingest/ enclave/ impl/src/ lib.rs {\tt enclave\_ init()}} when {\tt fog-ingest} enclaves start up and are stored in temporary memory (the {\tt ELRANGE}) until deprecated or the enclave is shut down.

    \item Local storage: {\tt Fog-ingest} enclaves encrypt ingress keys with an SGX Seal key when they are created, and save them on the local machine so they can be recovered if the enclave is restarted (only an enclave with the same {\tt MRENCLAVE} value on the same SGX-enabled machine can recover it).

    \item Peer enclaves: Ingress keys can be sent\marginnote{[MC-fog] fog/ingest/ server/src/ controller.rs {\tt sync\_keys\_ from\_re- mote()}} to `peer' enclaves (enclaves with the same {\tt MRENCLAVE} value), which act as back-ups in case the main enclave's machine has a critical failure.

    \item Last resort: In the worst case scenario where an ingress key is completely lost/unrecoverable, yet there exist blocks within that key's expected lifetime (i.e.\ younger than its creation and older than its expiration) whose outputs haven't been processed into {\tt ETxOutRecords} by that key, those `missed blocks' must be manually scanned by users for them to identify all owned outputs therein. The Fog system is designed so its operators can identify\marginnote{[MC-fog] fog/ingest/ client/src/ lib.rs {\tt report\_mis- sed\_block\_ range()}}[.2cm] missed blocks and send the outputs they contain to users.
\end{enumerate}



\section{Obtaining owned outputs from Fog}
\label{sec:fog-obtaining-outputs}

Obtaining owned outputs from Fog is fairly straightforward.

\begin{enumerate}
    \item Contact\marginnote{[MC-fog] fog/view/pr- otocol/src/ polling.rs {\tt poll()}} the {\tt fog-view} sub-service (it's not an enclave, just a normal server) and receive copies of all {\tt RngRecord}s containing public egress keys that were active over the range of blocks you wish to scan, along with any missed blocks in that range.\footnote{If your Fog service provider uses multiple `fog report ids' then you may receive a set of {\tt RngRecord}s corresponding to the report id in your public address (recall Section \ref{sec:fog-enabled-addresses}). In other words, a different set than might be received by a user with a different report id.} The {\tt fog-view} sub-service will also provide an {\tt AVR} for communicating with a {\tt fog-view} enclave that it owns.

    \item Using the public egress keys in those {\tt RngRecord}s\marginnote{[MC-fog] fog/view/pr- otocol/src/ user\_rng\_ set.rs {\tt ingest\_tx\_ out\_search\_ results()}}[.2cm], compute fog search keys with counters starting at 0. It is important to start by computing at least two fog search keys from each egress key (counter values `0' and `1').\footnote{We must use the `full' private view key $k^{v,i}$ instead of base key $k^v$, because {\tt RngRecords} use a shared secret between $k_{ingress} G$ and $K^{v,i}$.}\vspace{.155cm}
    \[FSK = \mathcal{H}(\textrm{algo version}, k^{v,i}*k_{egress} G, \textrm{counter})\textrm{.first\_16\_bytes}\]

    \item Send the fog search keys to the {\tt fog-view} enclave. This enclave has an OMAP loaded with {\tt ETxOutRecord}s passed to it by the Fog operator, where fog search keys are used as the hash map lookup keys (hence the name `fog search keys'). The enclave will look up\marginnote{*[MC-fog] fog/view/ enclave/ impl/src/ lib.rs {\tt query()}} any records corresponding to fog search keys sent by the user, and send them back.

    To prevent the Fog operator from inferring exactly how many outputs the user actually owns, the {\tt fog-view} enclave is designed to be constant-time and constant-bandwidth. In part, this means it will send back fake {\tt ETxOutRecord}s for any fog search keys that don't have entries in the OMAP. Note\marginnote{*[MC-fog] fog/view/ enclave/ impl/src/ e\_tx\_out\_ store.rs {\tt find\_re- cord()}}[.4cm] that the operator can still estimate the maximum number of outputs the user owns based on the overall volume of data transmitted between their {\tt fog-view} enclave and the user.

    \item Alongside {\tt ETxOutRecord}s returned to the user are `result codes' indicating if the records correspond with OMAP lookup matches or misses. If all returned records from a given fog-search-key-chain are matches (regardless of if they can be decrypted or used in any way) then it's possible the OMAP contains matches for higher counter values (which may be usable). The user must continue making requests\marginnote{[MC-fog] fog/view/pr- otocol/src/ polling.rs {\tt poll()}}[.6cm] until the enclave returns a `miss'.\footnote{If the {\tt fog-view} enclave's {\tt ETxOutRecord} OMAP is somehow corrupted, then it is conceivable for a `match' to exist for a higher counter value than a `miss'. However, there is no easy way for the user to detect this, so it is simplest to assume a `miss' ends the chain.}\footnote{Since the user may not know how many {\tt ETxOutRecords} they will ultimately receive, it may be useful to create progressively larger quantities of fog search keys with each request, to reduce the total number of requests made. In the MobileCoin implementation\marginnote{[MC-fog] fog/view/pr- otocol/src/ polling.rs {\tt poll()}} the request size is doubled with each iteration (starting at 2).}

    A user only stops making requests for a chain when not all records returned are matches. This is why the minimum chain-request size should be two. If the size was one it would be clear when a user doesn't own any outputs processed by a given egress key. By starting with two search keys, failure to make a second request means the user owns either one or zero outputs processed by the relevant egress key, which is ambiguous to the Fog operator.

    \item The user can decrypt\marginnote{[MC-fog] fog/view/pr- otocol/src/ user\_rng\_ set.rs {\tt ingest\_tx\_ out\_search\_ results()}} the {\tt ETxOutRecord}s he receives to reveal the outputs that he owns. Of course, he should also perform the usual steps for examining outputs to make sure he really owns them (computing the sender-receiver shared secrets, checking the one-time addresses [Section \ref{subsec:subaddresses-sending-to}], and decrypting the amounts [Section \ref{sec:pedersen-ringct}]).
\end{enumerate}

In the end, users obtain their owned outputs and Fog operators know few if any of the details (assuming either SGX secure enclaves are not breached, or Fog operators do not exploit any vulnerabilities that may be discovered).\footnote{If an SGX enclave breach does occur, then the Fog operator may be able to learn the private ingress key created by any {\tt fog-ingest} enclave with an exploitable SGX enclave version (ingress keys can be saved in anticipation of future breaches, and old enclave software (re)started at any time). A leaked private ingress key can decrypt all fog hints encrypted with that key, effectively revealing the true recipients of those hints' outputs. This risk can be avoided by manually scanning all on-chain outputs instead of using Fog, at the cost of time and downloading all those outputs.}



\section{Accessing blockchain information}
\label{sec:fog-accessing-blockchain-information}

In other RingCT-based cryptocurrencies such as Monero, users are only able to access blockchain data by running their own node (which provides them with a full copy of the blockchain), or making requests to node operators. Running a node is not a requirement all users are willing to accept, downloading large subsets of blockchain data is expensive and tedious, and requesting specific information from node operators is a privacy risk.

The Fog sub-service known as {\tt fog-ledger} is designed to alleviate those problems. In short, a {\tt fog-ledger} enclave can store OMAPs containing on-chain key images, a full and continuously-updated output Merkle tree for assembling membership proofs, and all on-chain outputs. Users can make direct requests to this enclave to privately learn information about the blockchain.\footnote{The {\tt fog-ledger} service is still in development at the time of writing this, so exact implementation details may differ from what is described to some extent. Readers should note that the {\tt fog-ledger} service prototype does not use OMAPs, so all queries are visible to the Fog operator.}


\subsection{Checking for spent outputs}
\label{subsec:fog-checking-spent-outputs}

The main Fog system is designed to connect users with their owned outputs, but has no way of knowing when those outputs are spent. Instead, output recipients must compute their outputs' key images, then check if those key images have appeared in the blockchain.

This is made simple by the {\tt fog-ledger} sub-service. Bob can send any key image to the {\tt fog-ledger} enclave, which will check if the key exists in its key image OMAP and inform Bob of the result.


\subsection{Acquiring membership proofs}
\label{subsec:fog-acquiring-membership-proofs}

To construct a new transaction, the transaction author must include membership proofs that show outputs spent by the transaction appear in the blockchain (recall Chapter \ref{chapter:membership-proofs}). These proofs can only be created from a full view of the blockchain.

MobileCoin has the ambitious design goal of making the on-chain transaction graph completely unknown to all observers. In keeping with that, obtaining membership proofs should not leak to Fog operators which ring members a user wishes to include in their transactions. To that end, the {\tt fog-ledger} enclave can maintain an OMAP of the continually-updating output Merkle tree. Requests for membership proofs at specific on-chain output indices can be resolved obliviously using that OMAP.


\subsection{Getting copies of outputs}
\label{subsec:fog-getting-copies-outputs}

How much needs to be said? The {\tt fog-ledger} enclave can store {\tt TxOuts} in an OMAP and deliver them to users on request (i.e.\ given on-chain output indices).
